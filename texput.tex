% Emacs, this is -*-latex-*-

\title{\href{https://github.com/vicente-gonzalez-ruiz/quantization}{Quantization}}

\author{Vicente González Ruiz}

\maketitle
\tableofcontents

\section{Definition}

\begin{figure}
  \svg{graphics/cuantif}{500}
  \caption{An example quantization procedure, applied to an analog
    signal.}
  \label{fig:cuantif}
\end{figure}

Quantization is the process of mapping a continuous range of values
into a finite range of discrete values. A
\href{https://en.wikipedia.org/wiki/Quantization_(signal_processing)}{\emph{quantizer}},
usually denoted by $Q$, is a system that:

\begin{enumerate}
\item In the analog case (see the Figure~\ref{fig:cuantif}), a
  \href{https://en.wikipedia.org/wiki/Pulse-amplitude_modulation}{PAM
    signal} $s(nT)$ (i.e. a sequence of analog samples
  where $s$ is an (analog) signal, $n\in{\mathbb{Z}}$ and $T$ is the
  sampling period) is transformed into a
  \href{https://en.wikipedia.org/wiki/Pulse-code_modulation}{PCM
    signal} ${\mathbf s}=s[n=0,1,\cdots]=$. Therefore, analog
  quantization is the process of mapping a continuous range of values
  (not necessarily countable) into a finite range of discrete values
  (necessarily countable)~\cite{vetterli1995wavelets}.

  Notice that, in general, analog signals are 1-dimensional, and that
  analog quantization is irreversible.
  
\item In the digital case, ${\mathbf s}$ is already a PCM signal (a
  sequence of integers of finite precision), the output of a quantizer
  is a sequence of quantization indexes ${\mathbf k}$, and the inverse
  system, called a \emph{dequantizer} or \emph{inverse quantizer}
  (denoted by $\text{Q}^{-1}$), can only recovers an approximated
  version of ${\mathbf s}$ that it will be denoted by $\tilde{\mathbf
    s}$.

  If we define the \emph{cardinality} operator $|\cdot|$ applied to a
  digital signal as the number of different values that such signal
  can take, i.e., the size of the signal alphabet, it ususally holds
  that
  \begin{equation}
    |{\mathbf s}|\leq|\tilde{\mathbf s}| = |{\mathbf k}|.
  \end{equation}
  As a consequence of this property, the values that ${\mathbf k}$,
  individually, can take will require less bits to be represented than
  the values that the original signal ${\mathbf s}$ can.

  Finally, note that $Q^{-1}$ is only a formal notation and does not
  correspond to the reciprocal function of $Q$ since quantization is
  noninvertible~\cite{duhamel2009joint} (Quantization is not
  \href{https://en.wikipedia.org/wiki/Linear_map}{linear} transform).
  
\end{enumerate}

\section{Quantization error}

Quantization is a lossy process that usually generates a distortion
between ${\mathbf s}$ and $\tilde{\mathbf s}$. The quantization error in unpredectible (for this reason is also called quantization noise), and therefore, we cannot recover ${\mathbf s}$.

We define the quantization error
\begin{equation}
  {\mathbf e} = {\mathbf s} - \tilde{{\mathbf s}},
\end{equation}
an the distortion is generally measured as the Mean Squared Error
determined by
\begin{equation}
  \text{MSE}({\mathbf s}, \tilde{\mathbf s}) = E(({\mathbf s} - \tilde{\mathbf s})^2),
  \label{eq:MSE}
\end{equation}
where $E(\cdot)$ denotes the
\href{https://en.wikipedia.org/wiki/Expected_value}{expectation}
operator.

\section{Quantizer design}

$Q$ should be designed to minimize the MSE, and this depends on how
the mapping between the input signal ${\mathbf s}$ and the quantized
signal $\tilde{\mathbf s}$ has been performed.

$Q$ is defined by a finite set of \emph{decision boundaries}
$\{{\mathbf d}_i; i\in {\mathbb{Z}}\}$ that describe a set or
cells\footnote{Also called ``bins''.} in the signal domain, and a
finite set of \emph{representation values} $\{{\mathbf r}_i; i\in
{\mathbb{Z}}\}$ (see Figure~\ref{fig:quantif}), both with the same
cardinality (because there is one representation level per cell). The
set ${\mathbf r}$ is called the \emph{codebook} and to their elements
${\mathbf r}_i$ \emph{codewords}.

Given a finite number $K$ of cells, to minimize the MSE, ${\mathbf d}$
and ${\mathbf r}$ are selected depending on the characteristics of
${\mathbf s}$. In general, we need to consider the statistical
distribution of the samples (or vectors of samples, depending on the
way we process the input) in ${\mathbf s}$.

Notice that $K$ has an impact on the output bit-rate of the quantizer
and therefore, we could be interested in minimizing the RD (Rate
Distortion) tradeoff instead of simply the MSE. However, such problem
in general is addressed by using an entropy codec at the output of
$Q$.

\section{Scalar quantization}

Scalar quantizers map each source sample independently from the other
samples and therefore, a quantization index ${\mathbf k}_i$ is
produced for each input sample ${\mathbf
  s}_i$~\cite{vruiz__scalar_quantization}.

\section{Vector quantization}

In vector quantization, several source samples are quantized
simultaneously and a single index is associated to a vector of source
samples. Vector quantization allows to account for the correlation
between source samples directly at the quantizer, which improves its
efficiency~\cite{duhamel2009joint}.

\section{Uniform quantization}

In uniform quantizers, the size of the cells is constant. For example,
in a scalar quantizer, the quantization step size is constant and
independent of the input signal.

Uniform quantizers are used in most A/D (analogic/digital) converters,
where it is expected the generation of uniformely distributed
sequences of samples. In the case of scalar uniform digital
quantizers, it is common to have $r_i=i$ and the input intervals are
of the form $(d_{i-1},d_i]=(i-1/2,i+1/2]$.

In the case of analog scalar uniform quantizer, if $R$ is the number
of bits of the quantizer per sample (for example, in the quantizer of
the Fig.~\ref{fig:cuantif}, $R=\lceil\log_2(5)\rceil=3$-bits and
$(Q=8)$), $\Delta$ decreases as $\frac{1}{2^R}$. Considering
Eq.~\ref{eq:MSE_uniform_scalar_quantizer} and
Eq.~\ref{eq:delta_definition}, we have that
\begin{equation}
  \text{MSE} = \frac{\Delta^2}{12} = \frac{(d_{\text max}-d_{\text
      min})^2}{12Q^2},
\end{equation}
where ${\mathbf d}_{\text max}$ and ${\mathbf d}_{\text min}$ are the
maximum and minimum decision boundary, respectively. Considering now
that $\sigma^2_{\mathbf s}=\frac{(d_{\text max}-d_{\text min})^2}{12}$
for uniform input PDF (Probability Density Function), we obtain that
\begin{equation}
  \text{MSE} = \sigma_{\mathbf s}^22^{-2R}.
\end{equation}

Now, if we add a bit to $R$, $R^{+1}=R+1$, then
\begin{equation*}
  \text{MSE}^{+1}=2^{-2(R+1)}\sigma_{\mathbf s} ^2 = 2^{-2}2^{-2R}\sigma_{\mathbf s}^2,
\end{equation*}
and
\begin{equation*}
  \text{SNR}^{+1} = 10\log_{10}\frac{\sigma_{\mathbf s}^2}{\text{MSE}^{+1}} = 10\log_{10}4\frac{\sigma_{\mathbf s}^2}{\text{MSE}} =
  10\log_{10} 4 + \text{SNR} \approx  6~\text{dB} + \text{SNR}.
\end{equation*}
This result is interesting because in a PCM system, the quality of
the signal is incremented $6$ dB with each bit. Notice that in
\href{https://en.wikipedia.org/wiki/High_fidelity}{HiFi}, the SNR
must be at least of $96$ dB, and
\begin{equation*}
  \frac{96}{6} = 16,
\end{equation*}
the number of bits per sample used in the
\href{https://en.wikipedia.org/wiki/Compact_disc}{Audio CDs}.

\section{Non-uniform quantization}

Depending on how the quantizer has been designed, it can be necesary to define the so called \emph{low and high overload
regions}, respectively, that in the case of analog quantization ban be described as
\begin{equation}
  s[n] = \{\begin{array}{ll}
  r_{\text{min}}, & \text{if $s(nT_s)<d_{\text{min}}$} \\
  r_{\text{max}}, & \text{if $s(nT_s)>d_{\text{max}}$} \\
  s(nT_s)+e(nT_s), & \text{otherwise}.
  \end{array}
\end{equation}
On other occasions, and depending on the PDF of ${\mathbf s}$, the
cells must be of different size to minimize the MSE. When all the
cells do not have the same size we have designed a non-uniform
quantizer.

\section{Resources}
\bibliography{quantization,coding}
